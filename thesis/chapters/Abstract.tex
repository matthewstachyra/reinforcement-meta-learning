/noindent Meta-learning, often referred to as "learning to learn," is a paradigm within machine learning 
that focuses on developing models that generalize knowledge across tasks. Such a model learns across 
tasks as it learns within tasks, enabling the model to adapt quickly to new, unseen tasks. A model for 
a new task can be generated with near-optimal parameters in no or very few training steps. Meta-learning 
in machine learning is desirable in cases where there is limited data for a task or when a model needs 
to learn efficiently within few data samples. This thesis investigates whether Meta-learning can be cast 
as Reinforcement learning (RL) problem for non-RL tasks. RL applied to Meta-learning has included 
hyperparameter search for neural architectures and meta-policies in RL task domains (Meta-Reinforcement 
Learning). RL has not been used to train models (parameter search) for non-RL tasks in the Meta-learning 
paradigm. This thesis asks: can Meta-learning be formulated as a sequential task? And if Meta-learning 
can be formulated as a sequential task, can Meta-learning be a discrete-time finite-horizon MDP solved 
by deep RL? Reinforcement Meta-Learning (REML) is proposed to formulate the Meta-learning task as 
composing neural networks layer by layer for sampled tasks with a shared parameter space across tasks. 
REML is evaluated according to the protocol proposed by Finn et al (2018) as used for the Model-agonistic 
meta-learning (MAML) algorithm. REML is first tested on regression tasks as a series of varying sinusoidal 
curves. Performance is evaluated looking at loss per step (convergence speed) for a model constructed by 
REML as compared to a model trained from scratch. REML is also tested on a few-shot learning task where 
it is given 5 and 10 samples and tested after 0, 1, and 10 gradient steps. Future work is to continue 
testing REML on classification and reinforcement learning tasks, and expanding  the action space of REML 
to decide hyperparameters in addition to parameters.
\clearpage